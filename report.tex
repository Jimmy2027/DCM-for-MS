\documentclass[a4paper,conference]{IEEEtran}
\usepackage[utf8]{inputenc}
\usepackage{csquotes}
\usepackage{multicol}
\usepackage{hyperref}
\hypersetup{
    colorlinks=true,
    linkcolor=blue,
    filecolor=magenta, 
    urlcolor=blue,
}
\urlstyle{same}

\title{DCM for MS}
%\author{%
%\IEEEauthorblockN{}
%\IEEEauthorblockA{ETH Z\"urich\\
%              D-ITET, ETH Zentrum\\
%              Email: @student.ethz.ch}
%
%\and
% Several authors with up to three affiliations:



%}


\date{}

\begin{document}
\maketitle

\begin{abstract}
The study of directed causal influences among brain regions, the effective connectivity (EC), plays an important role in understanding the brains operational principles \cite{friston_functional_2011} as well as its sicknesses. 
There exist multiple methods for EC analysis each coming with their advantages and disadvantages. 
In an attempt to compare these methods we re-evaluate a EC study of multiple sclerosis (MS) patients from Fleischer et al. \cite{fleischer} with a dynamic causal model (DCM) for cross spectral densities \cite{csd_for_dcm} and a regression DCM (rDCM) \cite{rdcm}.

\end{abstract}

\section{Introduction}

Multiple sclerosis manifests itself by introducing a change of connectivity of different brain regions in patients, which can be studied with effective connectivity (EC) – a model of the directed causal influences among brain regions. In a study, Fleischer et al. \cite{fleischer} compared serial longitudinal structural and resting-state fMRI of MS patients with healthy controls (HC) and reported that EC from the deep grey matter, frontal, prefrontal and temporal regions in the MS patients showed a continuous increase over the study period. To analyze EC, they performed two parallel analyses based on Causal Bayesian Networks (CBN) \cite{smith2011network} and on a Time-resolved Partial Directed Coherence (TPDC) method \cite{vergotte2017dynamics}.

In this study they decided not to use EC analyses based on dynamic causal modeling because a priori information on the existence of a connection between two regions is needed.
However, in order to show that this is not a limiting factor for the DCM, 
%(we expand their work by using...) 
we reproduce their results on their data using EC analyses based on dynamic causal modeling.

The parameters of a DCM comprise (1) parameters that mediate the influence of extrinsic inputs on the states, (2) parameters that mediate intrinsic coupling among the states, and (3) [bilinear] parameters that allow the inputs to modulate that coupling \cite{dcm}. While the parameters are usually estimated given known, deterministic inputs and responses of the observed system, we analyse resting state fMRI where the resting state inputs of the individual brain regions are unknown. Studies have shown that the resting state fMRI signals convey fluctuations in the low-frequency band of 0.01-0.08Hz \cite{csd_for_dcm, biswal1995functional, cordes2001frequencies}. This can be modeled using a DCM for cross spectral densities \cite{resting_sate_dcm} which unlike a stochastic DCM, tries to estimate the parameters of the cross correlation functions of the time varying fluctuations in neuronal states instead of the variations themselves. It does so by replacing the original time series with their second-order statistics, under stationary assumptions. This assumption makes this approach unable to model changes in EC caused by experimental manipulations. It is however able to deduce group differences in EC, e.g. EC differences between MS patients and HC.

\newpage
\begin{multicols}{2}


\section{Models and Methods}

\section{Results}

\section{Discussion}

\section{Summary}

\end{multicols}

\bibliography{references}
\bibliographystyle{ieeetr}

\end{document}